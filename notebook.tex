



    
\documentclass[11pt]{article}
    
    \usepackage{parskip}
    \setcounter{secnumdepth}{0} %Suppress section numbers
    \usepackage[breakable]{tcolorbox}
    \tcbset{nobeforeafter}
    \usepackage{needspace}
    \usepackage{minted}
    \usemintedstyle{jupyter_python}
    
    \usepackage[T1]{fontenc}
    % Nicer default font (+ math font) than Computer Modern for most use cases
    \usepackage{mathpazo}

    % Basic figure setup, for now with no caption control since it's done
    % automatically by Pandoc (which extracts ![](path) syntax from Markdown).
    \usepackage{graphicx}
    % We will generate all images so they have a width \maxwidth. This means
    % that they will get their normal width if they fit onto the page, but
    % are scaled down if they would overflow the margins.
    \makeatletter
    \def\maxwidth{\ifdim\Gin@nat@width>\linewidth\linewidth
    \else\Gin@nat@width\fi}
    \makeatother
    \let\Oldincludegraphics\includegraphics
    % Set max figure width to be 80% of text width, for now hardcoded.
    \renewcommand{\includegraphics}[1]{\Oldincludegraphics[width=.8\maxwidth]{#1}}
    % Ensure that by default, figures have no caption (until we provide a
    % proper Figure object with a Caption API and a way to capture that
    % in the conversion process - todo).
    \usepackage{caption}
    \DeclareCaptionLabelFormat{nolabel}{}
    \captionsetup{labelformat=nolabel}

    \usepackage{adjustbox} % Used to constrain images to a maximum size 
    \usepackage{xcolor} % Allow colors to be defined
    \usepackage{enumerate} % Needed for markdown enumerations to work
    \usepackage{geometry} % Used to adjust the document margins
    \usepackage{amsmath} % Equations
    \usepackage{amssymb} % Equations
    \usepackage{textcomp} % defines textquotesingle
    % Hack from http://tex.stackexchange.com/a/47451/13684:
    \AtBeginDocument{%
        \def\PYZsq{\textquotesingle}% Upright quotes in Pygmentized code
    }
    \usepackage{upquote} % Upright quotes for verbatim code
    \usepackage{eurosym} % defines \euro
    \usepackage[mathletters]{ucs} % Extended unicode (utf-8) support
    \usepackage[utf8x]{inputenc} % Allow utf-8 characters in the tex document
    \usepackage{fancyvrb} % verbatim replacement that allows latex
    \usepackage{grffile} % extends the file name processing of package graphics 
                         % to support a larger range 
    % The hyperref package gives us a pdf with properly built
    % internal navigation ('pdf bookmarks' for the table of contents,
    % internal cross-reference links, web links for URLs, etc.)
    \usepackage{hyperref}
    \usepackage{longtable} % longtable support required by pandoc >1.10
    \usepackage{booktabs}  % table support for pandoc > 1.12.2
    \usepackage[inline]{enumitem} % IRkernel/repr support (it uses the enumerate* environment)
    \usepackage[normalem]{ulem} % ulem is needed to support strikethroughs (\sout)
                                % normalem makes italics be italics, not underlines
    

    \let\Oldtex\TeX     % provide compatibility with nbconvert <= 5.3.1
    \let\Oldlatex\LaTeX % pre-included in nbconvert > 5.3.1 so redundant
    
    % Colors for the hyperref package
    \definecolor{urlcolor}{rgb}{0,.145,.698}
    \definecolor{linkcolor}{rgb}{.71,0.21,0.01}
    \definecolor{citecolor}{rgb}{.12,.54,.11}

    % ANSI colors
    \definecolor{ansi-black}{HTML}{3E424D}
    \definecolor{ansi-black-intense}{HTML}{282C36}
    \definecolor{ansi-red}{HTML}{E75C58}
    \definecolor{ansi-red-intense}{HTML}{B22B31}
    \definecolor{ansi-green}{HTML}{00A250}
    \definecolor{ansi-green-intense}{HTML}{007427}
    \definecolor{ansi-yellow}{HTML}{DDB62B}
    \definecolor{ansi-yellow-intense}{HTML}{B27D12}
    \definecolor{ansi-blue}{HTML}{208FFB}
    \definecolor{ansi-blue-intense}{HTML}{0065CA}
    \definecolor{ansi-magenta}{HTML}{D160C4}
    \definecolor{ansi-magenta-intense}{HTML}{A03196}
    \definecolor{ansi-cyan}{HTML}{60C6C8}
    \definecolor{ansi-cyan-intense}{HTML}{258F8F}
    \definecolor{ansi-white}{HTML}{C5C1B4}
    \definecolor{ansi-white-intense}{HTML}{A1A6B2}

    % commands and environments needed by pandoc snippets
    % extracted from the output of `pandoc -s`
    \providecommand{\tightlist}{%
      \setlength{\itemsep}{0pt}\setlength{\parskip}{0pt}}
    \DefineVerbatimEnvironment{Highlighting}{Verbatim}{commandchars=\\\{\}}
    % Add ',fontsize=\small' for more characters per line
    \newenvironment{Shaded}{}{}
    \newcommand{\KeywordTok}[1]{\textcolor[rgb]{0.00,0.44,0.13}{\textbf{{#1}}}}
    \newcommand{\DataTypeTok}[1]{\textcolor[rgb]{0.56,0.13,0.00}{{#1}}}
    \newcommand{\DecValTok}[1]{\textcolor[rgb]{0.25,0.63,0.44}{{#1}}}
    \newcommand{\BaseNTok}[1]{\textcolor[rgb]{0.25,0.63,0.44}{{#1}}}
    \newcommand{\FloatTok}[1]{\textcolor[rgb]{0.25,0.63,0.44}{{#1}}}
    \newcommand{\CharTok}[1]{\textcolor[rgb]{0.25,0.44,0.63}{{#1}}}
    \newcommand{\StringTok}[1]{\textcolor[rgb]{0.25,0.44,0.63}{{#1}}}
    \newcommand{\CommentTok}[1]{\textcolor[rgb]{0.38,0.63,0.69}{\textit{{#1}}}}
    \newcommand{\OtherTok}[1]{\textcolor[rgb]{0.00,0.44,0.13}{{#1}}}
    \newcommand{\AlertTok}[1]{\textcolor[rgb]{1.00,0.00,0.00}{\textbf{{#1}}}}
    \newcommand{\FunctionTok}[1]{\textcolor[rgb]{0.02,0.16,0.49}{{#1}}}
    \newcommand{\RegionMarkerTok}[1]{{#1}}
    \newcommand{\ErrorTok}[1]{\textcolor[rgb]{1.00,0.00,0.00}{\textbf{{#1}}}}
    \newcommand{\NormalTok}[1]{{#1}}
    
    % Additional commands for more recent versions of Pandoc
    \newcommand{\ConstantTok}[1]{\textcolor[rgb]{0.53,0.00,0.00}{{#1}}}
    \newcommand{\SpecialCharTok}[1]{\textcolor[rgb]{0.25,0.44,0.63}{{#1}}}
    \newcommand{\VerbatimStringTok}[1]{\textcolor[rgb]{0.25,0.44,0.63}{{#1}}}
    \newcommand{\SpecialStringTok}[1]{\textcolor[rgb]{0.73,0.40,0.53}{{#1}}}
    \newcommand{\ImportTok}[1]{{#1}}
    \newcommand{\DocumentationTok}[1]{\textcolor[rgb]{0.73,0.13,0.13}{\textit{{#1}}}}
    \newcommand{\AnnotationTok}[1]{\textcolor[rgb]{0.38,0.63,0.69}{\textbf{\textit{{#1}}}}}
    \newcommand{\CommentVarTok}[1]{\textcolor[rgb]{0.38,0.63,0.69}{\textbf{\textit{{#1}}}}}
    \newcommand{\VariableTok}[1]{\textcolor[rgb]{0.10,0.09,0.49}{{#1}}}
    \newcommand{\ControlFlowTok}[1]{\textcolor[rgb]{0.00,0.44,0.13}{\textbf{{#1}}}}
    \newcommand{\OperatorTok}[1]{\textcolor[rgb]{0.40,0.40,0.40}{{#1}}}
    \newcommand{\BuiltInTok}[1]{{#1}}
    \newcommand{\ExtensionTok}[1]{{#1}}
    \newcommand{\PreprocessorTok}[1]{\textcolor[rgb]{0.74,0.48,0.00}{{#1}}}
    \newcommand{\AttributeTok}[1]{\textcolor[rgb]{0.49,0.56,0.16}{{#1}}}
    \newcommand{\InformationTok}[1]{\textcolor[rgb]{0.38,0.63,0.69}{\textbf{\textit{{#1}}}}}
    \newcommand{\WarningTok}[1]{\textcolor[rgb]{0.38,0.63,0.69}{\textbf{\textit{{#1}}}}}
    
    
    % Define a nice break command that doesn't care if a line doesn't already
    % exist.
    \def\br{\hspace*{\fill} \\* }
    % Math Jax compatability definitions
    \def\gt{>}
    \def\lt{<}
    % Document parameters
    \title{Assignment 1}
    
    
    
% Pygments definitions
    
    \makeatletter
    \newcommand*\@iflatexlater{\@ifl@t@r\fmtversion}
    \@iflatexlater{2016/03/01}{
	    \newcommand{\wordboundary}{4095}}{
	    \newcommand{\wordboundary}{255}}
    \makeatother

    \newif\ifcode
    \codefalse
    \definecolor{Grey}{rgb}{0.40,0.40,0.40}
    %If using XeLaTeX, use magic to not highlight . operators with purple.
    \ifdefined\XeTeXcharclass
    \XeTeXinterchartokenstate = 1
    \newXeTeXintercharclass \mycharclassGrey
    \XeTeXcharclass `. \mycharclassGrey
    \XeTeXinterchartoks 0 \mycharclassGrey   = {\bgroup\ifcode\color{Grey}\else\fi}

    \XeTeXinterchartoks \wordboundary \mycharclassGrey = {\bgroup\ifcode\color{Grey}\else\fi}

    \XeTeXinterchartoks \mycharclassGrey 0   = {\egroup}
    \XeTeXinterchartoks \mycharclassGrey \wordboundary = {\egroup}
    \fi %end magical operator highlighting
    %End Reconfigured Pygments
    
   
    % Exact colors from NB
    \definecolor{incolor}{HTML}{303F9F}
    \definecolor{outcolor}{HTML}{D84315}
    \definecolor{cellborder}{HTML}{CFCFCF}
    \definecolor{cellbackground}{HTML}{F7F7F7}

    % needed definitions
    \newif\ifleftmargins
    \newlength{\promptlength}

    % cell style settings
        \leftmarginsfalse

    
    % prompt
    \newcommand{\prompt}[3]{
        \needspace{1.1cm}
        \settowidth{\promptlength}{ #1 [#3] }
        \ifleftmargins\hspace{-\promptlength}\hspace{-5pt}\fi
        {\color{#2}#1 [#3]:}
        \ifleftmargins\vspace{-2.7ex}\fi
    }
    
    
    % environments
    \newenvironment{OutVerbatim}{\VerbatimEnvironment%
        \begin{tcolorbox}[breakable, boxrule=.5pt, size=fbox, pad at break*=1mm, opacityfill=0]
            \begin{Verbatim}
            }{
            \end{Verbatim}
        \end{tcolorbox}
    }
    
    %Updated MathJax Compatibility (if future behaviour of the notebook changes this may be removed)
    \renewcommand{\TeX}{\ifmmode \textrm{\Oldtex} \else \textbackslash TeX \fi}
    \renewcommand{\LaTeX}{\ifmmode \Oldlatex \else \textbackslash LaTeX \fi}
    
    % Header Adjustments
    \renewcommand{\paragraph}{\textbf}
    \renewcommand{\subparagraph}[1]{\textit{\textbf{#1}}}

    
    % Prevent overflowing lines due to hard-to-break entities
    \sloppy 
    % Setup hyperref package
    \hypersetup{
      breaklinks=true,  % so long urls are correctly broken across lines
      colorlinks=true,
      urlcolor=urlcolor,
      linkcolor=linkcolor,
      citecolor=citecolor,
      }
    % Slightly bigger margins than the latex defaults
    \geometry{verbose,tmargin=.5in,bmargin=.7in,lmargin=.5in,rmargin=.5in}
    

    \begin{document}
    
    
    
    
    

    
    \hypertarget{assignment-1}{%
\section{Assignment 1}\label{assignment-1}}

    
\prompt{In}{incolor}{3}
\codetrue
\begin{tcolorbox}[breakable, size=fbox, boxrule=1pt, pad at break*=1mm, colback=cellbackground, colframe=cellborder]
\begin{minted}[breaklines=True]{ipython3}
import sympy as sym 
import sympy.vector
from sympy import Derivative, symbols, cos, sin, simplify, sqrt, Matrix
from sympy.physics.mechanics import *
init_vprinting()
\end{minted}
\end{tcolorbox}
\codefalse

    
\prompt{In}{incolor}{4}
\codetrue
\begin{tcolorbox}[breakable, size=fbox, boxrule=1pt, pad at break*=1mm, colback=cellbackground, colframe=cellborder]
\begin{minted}[breaklines=True]{ipython3}
theta, beta, gamma, theta_M = symbols('theta beta gamma theta_M')
\end{minted}
\end{tcolorbox}
\codefalse

    \hypertarget{question-1}{%
\subsection{Question 1}\label{question-1}}

\hypertarget{b}{%
\subsubsection{b)}\label{b}}

Let the variables with ``M'' subscripts denote quantities in the moving
frame.

\(dN = n(\theta, \phi) d\Omega\)

\(dN_M = n_M(\theta_M, \phi_M) d\Omega_M\)

Where:

\(d\Omega = \sin(\theta) d\theta d\phi\)

\(d\Omega_M = \sin(\theta_M) d\theta_M d\phi_M\)

\(\theta\) is the angle light makes with the x-axis in the stationary
frame.

\(\theta_M\) is the angle light makes with the x-axis in the moving
frame.

    \hypertarget{here-is-what-we-know}{%
\paragraph{Here is what we know:}\label{here-is-what-we-know}}

\(n(\theta, \phi) = Constant\) \(\because\) Stars are isotropic in
stationary frame

\(N = N_M\) \(\because\) Total amount of start seen by both observers
have to be the same.

\(\therefore dN = dN_M\)

We define \(\theta\) and \(\theta_M\) to be the polar angle relative to
the \(x\) and \(x_M\) axis, respectively. The azimuthal angle \(\phi\)
is in the stationary \(yz\)-plane, and \(\phi_M\) is in the moving
\(y_Mz_M\)-plane. As the moving observer is moving in the \(\hat{x}\)
direction, there will be contraction along the \(\hat{x}\) direction;
therefore, \(\theta \neq \theta_M\). The \(yz\)-plane is unaffected,
thus \(\phi = \phi_M\).

\(\phi = \phi_M \Rightarrow d\phi = d\phi_M\)

    Taking the integral of \(dN\) to find \(N\):

\(N = \int n(\theta, \phi) d\Omega = n(\theta, \phi) \int d\Omega = n(\theta, \phi) \int_0^{2\pi} \int_0^{\pi} \sin(\theta) d\theta d\phi = 4\pi n(\theta, \phi)\)

Now we find \(n_M(\theta_M, \phi_M)\):

\(n_M(\theta_M, \phi_M) = \frac{dN_M}{d\Omega_M} = \frac{dN_M}{d\Omega} \frac{d\Omega}{d\Omega_M} = \frac{dN}{d\Omega} \frac{d\Omega}{d\Omega_M} = \frac{N}{4\pi} \frac{d\Omega}{d\Omega_M} = \frac{N}{4\pi} \frac{d\Omega}{d\phi} \frac{d\phi_M}{d\Omega_M} = \frac{N}{4\pi} \frac{\sin(\theta) d\theta}{\sin(\theta_M)d\theta_M}\)

    We know:

\(\cos(\theta_M) = \frac{\cos(\theta) - \beta}{1 - \beta \cos(\theta)}\)

\(\therefore \frac{d}{d \theta_M} \cos(\theta_M) = \frac{d}{d \theta} \frac{\cos(\theta) - \beta}{1 - \beta \cos(\theta)}\)

However, this is what the resting observer calculates for the moving
observer in the basis of the rest frame.

Shifting to the moving frame, the angle of light recieved by the moving
observer:

\(\cos(\theta) = \frac{\cos(\theta_M) - \beta}{1 - \beta \cos(\theta_M)}\)

\(\therefore \frac{d}{d \theta} \cos(\theta) = \frac{d}{d \theta_M} \frac{\cos(\theta_M) - \beta}{1 - \beta \cos(\theta_M)}\)

    \hypertarget{find-the-derivatives}{%
\paragraph{Find the Derivatives}\label{find-the-derivatives}}

\(\frac{d}{d \theta} \cos(\theta) =\)

    
\prompt{In}{incolor}{5}
\codetrue
\begin{tcolorbox}[breakable, size=fbox, boxrule=1pt, pad at break*=1mm, colback=cellbackground, colframe=cellborder]
\begin{minted}[breaklines=True]{ipython3}
xmove = cos(theta)
dxmove = Derivative(xmove, theta).doit()
dxmove
\end{minted}
\end{tcolorbox}
\codefalse
 
            
\prompt{Out}{outcolor}{5}
    
    $$- \operatorname{sin}\left(\theta\right)$$

    

    \(\frac{d}{d\theta} \frac{\cos(\theta) - \beta}{1 - \beta \cos(\theta)} =\)

    
\prompt{In}{incolor}{6}
\codetrue
\begin{tcolorbox}[breakable, size=fbox, boxrule=1pt, pad at break*=1mm, colback=cellbackground, colframe=cellborder]
\begin{minted}[breaklines=True]{ipython3}
xstat = (cos(theta_M) - beta)/(1 - beta*cos(theta_M))
dxstat = Derivative(xstat, theta_M).doit()
dxstat.simplify() 
\end{minted}
\end{tcolorbox}
\codefalse
 
            
\prompt{Out}{outcolor}{6}
    
    $$\frac{\left(\beta^{2} - 1\right) \operatorname{sin}\left(\theta_{M}\right)}{\left(\beta \operatorname{cos}\left(\theta_{M}\right) - 1\right)^{2}}$$

    

    \$ \therefore \sin(\theta) d\theta =
\frac{1 - \beta^2 }{(\beta\cos(\theta_M) - 1)^2} \sin(\theta\_M)
d\theta\_M \Rightarrow
\frac{\sin(\theta) d\theta}{\sin(\theta_M) d\theta_M} =
\frac{1 - \beta^2 }{(1 - \beta\cos(\theta_M))^2} \$

    Combining this with what we got for \(n_M(\theta_M, \phi_M)\):

\(n_M(\theta_M, \phi_M) = \frac{N}{4\pi} \frac{d\Omega}{d\phi} \frac{d\phi_M}{d\Omega_M} = \frac{\sin(\theta) d\theta}{\sin(\theta_M)d\theta_M}\)

We have

\(n_M(\theta_M, \phi_M) = \frac{N}{4\pi} \frac{d\Omega}{d\phi} \frac{d\phi_M}{d\Omega_M} = \frac{N}{4\pi} \frac{\sin(\theta) d\theta}{\sin(\theta_M)d\theta_M} = \frac{N}{4\pi} \left[ \frac{1 - \beta^2 }{(1-\beta\cos(\theta_M))^2} \right]\)

    \hypertarget{c}{%
\subsubsection{c)}\label{c}}

As \(\beta \rightarrow 1\):

\(1-\beta^2 \rightarrow 0\)

\(1-\beta \cos(\theta_M) \rightarrow 1-\cos(\theta_M)\)

\(\therefore \frac{N}{4\pi} \left[ \frac{1 - \beta^2 }{(1-\beta\cos(\theta))^2} \right] \rightarrow 0\)

    This means all the stars in the sky has disappeared. If we look at the
\(1-\beta \cos(\theta)\) term we can get a sense of where they went.
\(1-\beta \cos(\theta)\) has a minimum value of \(-1\) at \(\theta = 0\)
and a maximum value of \(1\) at \(\theta = \pi\), so when
\(\theta = 0\), \(n_m(\theta_M, \phi_M)\) goes to infinity. This tells
us all the stars are ``moving'' behind the moving observer as the moving
observer approaches the speed of light. At the the speed of light, all
the stars have condensed into a single point behind the moving observer.

    \hypertarget{question-2}{%
\subsection{Question 2}\label{question-2}}

    
\prompt{In}{incolor}{4}
\codetrue
\begin{tcolorbox}[breakable, size=fbox, boxrule=1pt, pad at break*=1mm, colback=cellbackground, colframe=cellborder]
\begin{minted}[breaklines=True]{ipython3}
c = symbols('c')

C = sympy.vector.CoordSys3D('C')
E1, E2, E3 = symbols('E1:4')
E_vec = E1*C.i+E2*C.j+E3*C.k

B1, B2, B3 = symbols('B1:4')

v1, v2, v3 = symbols('v1:4')
v_vec = v1*C.i+v2*C.j+v3*C.k
v = sqrt(v1**2+v2**2+v3**2)
\end{minted}
\end{tcolorbox}
\codefalse

    Find out what this term looks like:

\$ \vec{B'} = -\frac{\gamma}{c^2} (\vec{v} \times \vec{E}) \$

    
\prompt{In}{incolor}{7}
\codetrue
\begin{tcolorbox}[breakable, size=fbox, boxrule=1pt, pad at break*=1mm, colback=cellbackground, colframe=cellborder]
\begin{minted}[breaklines=True]{ipython3}
B_cross = -gamma/c**2*v_vec.cross(E_vec)
display(B_cross)
\end{minted}
\end{tcolorbox}
\codefalse

    $$(- \frac{\gamma \left(- E_{2} v_{3} + E_{3} v_{2}\right)}{c^{2}})\mathbf{\hat{i}_{C}} + (- \frac{\gamma \left(E_{1} v_{3} - E_{3} v_{1}\right)}{c^{2}})\mathbf{\hat{j}_{C}} + (- \frac{\gamma \left(- E_{1} v_{2} + E_{2} v_{1}\right)}{c^{2}})\mathbf{\hat{k}_{C}}$$

    
    Here we define the various tensors:

\(F_{\mu \nu}\) (F) is the tensor for the electromagnetic field

\(F_{\mu \nu 0}\) (F0) is the tensor for the electromagnetic field in a
frame where \(\vec{B} = 0\)

\(\Lambda_{\mu'}^{\mu}\) (Lam) is the tensor that transforms F to F'

\$ F'\emph{\{\mu' \nu`\} = \Lambda\emph{\{\mu'\}\^{}\{\mu\}
\Lambda}\{\nu'\}\^{}\{\nu\} F}\{\mu \nu\} \$

    
\prompt{In}{incolor}{13}
\codetrue
\begin{tcolorbox}[breakable, size=fbox, boxrule=1pt, pad at break*=1mm, colback=cellbackground, colframe=cellborder]
\begin{minted}[breaklines=True]{ipython3}
F = Matrix([[0, E1, E2, E3], 
            [-E1, 0, -c*B3, c*B2], 
            [-E2, c*B3, 0, -c*B1], 
            [-E3, -c*B2, c*B1, 0]])/c

F0 = F.subs({B1:0, B2:0, B3:0})

Lam = Matrix([[gamma, v1*gamma/c, v2*gamma/c, v3*gamma/c], 
              [v1*gamma/c, 1+(gamma-1)*v1**2/v**2, (gamma-1)*v1*v2/v**2, (gamma-1)*v1*v3/v**2],
              [gamma*v2/c, (gamma-1)*v2*v1/v**2, 1+(gamma-1)*v2**2/v**2, (gamma-1)*v2*v3/v**2],
              [gamma*v3/c, (gamma-1)*v3*v1/v**2, (gamma-1)*v3*v2/v**2, 1+(gamma-1)*v3**2/v**2]])
F, F0, Lam
\end{minted}
\end{tcolorbox}
\codefalse
 
            
\prompt{Out}{outcolor}{13}
    
    $$\left ( \left[\begin{matrix}0 & \frac{E_{1}}{c} & \frac{E_{2}}{c} & \frac{E_{3}}{c}\\- \frac{E_{1}}{c} & 0 & - B_{3} & B_{2}\\- \frac{E_{2}}{c} & B_{3} & 0 & - B_{1}\\- \frac{E_{3}}{c} & - B_{2} & B_{1} & 0\end{matrix}\right], \quad \left[\begin{matrix}0 & \frac{E_{1}}{c} & \frac{E_{2}}{c} & \frac{E_{3}}{c}\\- \frac{E_{1}}{c} & 0 & 0 & 0\\- \frac{E_{2}}{c} & 0 & 0 & 0\\- \frac{E_{3}}{c} & 0 & 0 & 0\end{matrix}\right], \quad \left[\begin{matrix}\gamma & \frac{\gamma v_{1}}{c} & \frac{\gamma v_{2}}{c} & \frac{\gamma v_{3}}{c}\\\frac{\gamma v_{1}}{c} & \frac{v_{1}^{2} \left(\gamma - 1\right)}{v_{1}^{2} + v_{2}^{2} + v_{3}^{2}} + 1 & \frac{v_{1} v_{2} \left(\gamma - 1\right)}{v_{1}^{2} + v_{2}^{2} + v_{3}^{2}} & \frac{v_{1} v_{3} \left(\gamma - 1\right)}{v_{1}^{2} + v_{2}^{2} + v_{3}^{2}}\\\frac{\gamma v_{2}}{c} & \frac{v_{1} v_{2} \left(\gamma - 1\right)}{v_{1}^{2} + v_{2}^{2} + v_{3}^{2}} & \frac{v_{2}^{2} \left(\gamma - 1\right)}{v_{1}^{2} + v_{2}^{2} + v_{3}^{2}} + 1 & \frac{v_{2} v_{3} \left(\gamma - 1\right)}{v_{1}^{2} + v_{2}^{2} + v_{3}^{2}}\\\frac{\gamma v_{3}}{c} & \frac{v_{1} v_{3} \left(\gamma - 1\right)}{v_{1}^{2} + v_{2}^{2} + v_{3}^{2}} & \frac{v_{2} v_{3} \left(\gamma - 1\right)}{v_{1}^{2} + v_{2}^{2} + v_{3}^{2}} & \frac{v_{3}^{2} \left(\gamma - 1\right)}{v_{1}^{2} + v_{2}^{2} + v_{3}^{2}} + 1\end{matrix}\right]\right )$$

    

    Now we find \$ F'\emph{\{\mu' \nu`\} = \Lambda\emph{\{\mu'\}\^{}\{\mu\}
\Lambda}\{\nu'\}\^{}\{\nu\} F}\{\mu \nu\} \$

    
\prompt{In}{incolor}{14}
\codetrue
\begin{tcolorbox}[breakable, size=fbox, boxrule=1pt, pad at break*=1mm, colback=cellbackground, colframe=cellborder]
\begin{minted}[breaklines=True]{ipython3}
def f(i, j):
    a = 0
    for nu in range(4):
        for mu in range(4):
            a += Lam[i,mu]*Lam[nu,j]*F0[mu,nu]
    return(a)

F_prime = Matrix(4,4,f)

for i in range(4):
    for n in range(4):
        F_prime[i, n] = F_prime[i, n].simplify()
        
display(F_prime)
\end{minted}
\end{tcolorbox}
\codefalse

    $$\left[\begin{matrix}0 & \frac{\gamma \left(c^{2} \left(E_{1} \left(v_{1}^{2} \left(\gamma - 1\right) + v_{1}^{2} + v_{2}^{2} + v_{3}^{2}\right) + E_{2} v_{1} v_{2} \left(\gamma - 1\right) + E_{3} v_{1} v_{3} \left(\gamma - 1\right)\right) - \gamma v_{1} \left(v_{1}^{2} + v_{2}^{2} + v_{3}^{2}\right) \left(E_{1} v_{1} + E_{2} v_{2} + E_{3} v_{3}\right)\right)}{c^{3} \left(v_{1}^{2} + v_{2}^{2} + v_{3}^{2}\right)} & \frac{\gamma \left(c^{2} \left(E_{1} v_{1} v_{2} \left(\gamma - 1\right) + E_{2} \left(v_{1}^{2} + v_{2}^{2} \left(\gamma - 1\right) + v_{2}^{2} + v_{3}^{2}\right) + E_{3} v_{2} v_{3} \left(\gamma - 1\right)\right) - \gamma v_{2} \left(v_{1}^{2} + v_{2}^{2} + v_{3}^{2}\right) \left(E_{1} v_{1} + E_{2} v_{2} + E_{3} v_{3}\right)\right)}{c^{3} \left(v_{1}^{2} + v_{2}^{2} + v_{3}^{2}\right)} & \frac{\gamma \left(c^{2} \left(E_{1} v_{1} v_{3} \left(\gamma - 1\right) + E_{2} v_{2} v_{3} \left(\gamma - 1\right) + E_{3} \left(v_{1}^{2} + v_{2}^{2} + v_{3}^{2} \left(\gamma - 1\right) + v_{3}^{2}\right)\right) - \gamma v_{3} \left(v_{1}^{2} + v_{2}^{2} + v_{3}^{2}\right) \left(E_{1} v_{1} + E_{2} v_{2} + E_{3} v_{3}\right)\right)}{c^{3} \left(v_{1}^{2} + v_{2}^{2} + v_{3}^{2}\right)}\\\frac{\gamma \left(- c^{2} \left(E_{1} \left(v_{1}^{2} \left(\gamma - 1\right) + v_{1}^{2} + v_{2}^{2} + v_{3}^{2}\right) + E_{2} v_{1} v_{2} \left(\gamma - 1\right) + E_{3} v_{1} v_{3} \left(\gamma - 1\right)\right) + \gamma v_{1} \left(v_{1}^{2} + v_{2}^{2} + v_{3}^{2}\right) \left(E_{1} v_{1} + E_{2} v_{2} + E_{3} v_{3}\right)\right)}{c^{3} \left(v_{1}^{2} + v_{2}^{2} + v_{3}^{2}\right)} & 0 & \frac{\gamma \left(- E_{1} v_{2} + E_{2} v_{1}\right)}{c^{2}} & \frac{\gamma \left(- E_{1} v_{3} + E_{3} v_{1}\right)}{c^{2}}\\\frac{\gamma \left(- c^{2} \left(E_{1} v_{1} v_{2} \left(\gamma - 1\right) + E_{2} \left(v_{1}^{2} + v_{2}^{2} \left(\gamma - 1\right) + v_{2}^{2} + v_{3}^{2}\right) + E_{3} v_{2} v_{3} \left(\gamma - 1\right)\right) + \gamma v_{2} \left(v_{1}^{2} + v_{2}^{2} + v_{3}^{2}\right) \left(E_{1} v_{1} + E_{2} v_{2} + E_{3} v_{3}\right)\right)}{c^{3} \left(v_{1}^{2} + v_{2}^{2} + v_{3}^{2}\right)} & \frac{\gamma \left(E_{1} v_{2} - E_{2} v_{1}\right)}{c^{2}} & 0 & \frac{\gamma \left(- E_{2} v_{3} + E_{3} v_{2}\right)}{c^{2}}\\\frac{\gamma \left(- c^{2} \left(E_{1} v_{1} v_{3} \left(\gamma - 1\right) + E_{2} v_{2} v_{3} \left(\gamma - 1\right) + E_{3} \left(v_{1}^{2} + v_{2}^{2} + v_{3}^{2} \left(\gamma - 1\right) + v_{3}^{2}\right)\right) + \gamma v_{3} \left(v_{1}^{2} + v_{2}^{2} + v_{3}^{2}\right) \left(E_{1} v_{1} + E_{2} v_{2} + E_{3} v_{3}\right)\right)}{c^{3} \left(v_{1}^{2} + v_{2}^{2} + v_{3}^{2}\right)} & \frac{\gamma \left(E_{1} v_{3} - E_{3} v_{1}\right)}{c^{2}} & \frac{\gamma \left(E_{2} v_{3} - E_{3} v_{2}\right)}{c^{2}} & 0\end{matrix}\right]$$

    
    Pick out the values for \(\vec{B}\):

    
\prompt{In}{incolor}{15}
\codetrue
\begin{tcolorbox}[breakable, size=fbox, boxrule=1pt, pad at break*=1mm, colback=cellbackground, colframe=cellborder]
\begin{minted}[breaklines=True]{ipython3}
B_ten = (F_prime[3,2]*C.i - F_prime[3,1]*C.j + F_prime[2,1]*C.k).simplify()
display(B_ten)
\end{minted}
\end{tcolorbox}
\codefalse

    $$(\frac{\gamma \left(E_{2} v_{3} - E_{3} v_{2}\right)}{c^{2}})\mathbf{\hat{i}_{C}} + (\frac{\gamma \left(- E_{1} v_{3} + E_{3} v_{1}\right)}{c^{2}})\mathbf{\hat{j}_{C}} + (\frac{\gamma \left(E_{1} v_{2} - E_{2} v_{1}\right)}{c^{2}})\mathbf{\hat{k}_{C}}$$

    
    Check if \$\vec{B'} + \frac{\gamma}{c^2} (\vec{v} \times \vec{E}) = 0 \$

    
\prompt{In}{incolor}{16}
\codetrue
\begin{tcolorbox}[breakable, size=fbox, boxrule=1pt, pad at break*=1mm, colback=cellbackground, colframe=cellborder]
\begin{minted}[breaklines=True]{ipython3}
result = B_ten - B_cross 
result.simplify()
\end{minted}
\end{tcolorbox}
\codefalse
 
            
\prompt{Out}{outcolor}{16}
    
    $$\mathbf{\hat{0}}$$

    

    \(\vec{B'} + \frac{\gamma}{c^2} (\vec{v} \times \vec{E}) = 0\)

\(\therefore \vec{B'} = -\frac{\gamma}{c^2} (\vec{v} \times \vec{E})\)


    % Add a bibliography block to the postdoc
    
    
    
    \end{document}
